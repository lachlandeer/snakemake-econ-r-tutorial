\documentclass[]{book}
\usepackage{lmodern}
\usepackage{amssymb,amsmath}
\usepackage{ifxetex,ifluatex}
\usepackage{fixltx2e} % provides \textsubscript
\ifnum 0\ifxetex 1\fi\ifluatex 1\fi=0 % if pdftex
  \usepackage[T1]{fontenc}
  \usepackage[utf8]{inputenc}
\else % if luatex or xelatex
  \ifxetex
    \usepackage{mathspec}
  \else
    \usepackage{fontspec}
  \fi
  \defaultfontfeatures{Ligatures=TeX,Scale=MatchLowercase}
\fi
% use upquote if available, for straight quotes in verbatim environments
\IfFileExists{upquote.sty}{\usepackage{upquote}}{}
% use microtype if available
\IfFileExists{microtype.sty}{%
\usepackage{microtype}
\UseMicrotypeSet[protrusion]{basicmath} % disable protrusion for tt fonts
}{}
\usepackage[margin=1in]{geometry}
\usepackage{hyperref}
\hypersetup{unicode=true,
            pdftitle={Reproducible Research Workflows with Snakemake and R},
            pdfauthor={Lachlan Deer; Julian Langer},
            pdfborder={0 0 0},
            breaklinks=true}
\urlstyle{same}  % don't use monospace font for urls
\usepackage{natbib}
\bibliographystyle{apalike}
\usepackage{color}
\usepackage{fancyvrb}
\newcommand{\VerbBar}{|}
\newcommand{\VERB}{\Verb[commandchars=\\\{\}]}
\DefineVerbatimEnvironment{Highlighting}{Verbatim}{commandchars=\\\{\}}
% Add ',fontsize=\small' for more characters per line
\usepackage{framed}
\definecolor{shadecolor}{RGB}{248,248,248}
\newenvironment{Shaded}{\begin{snugshade}}{\end{snugshade}}
\newcommand{\KeywordTok}[1]{\textcolor[rgb]{0.13,0.29,0.53}{\textbf{{#1}}}}
\newcommand{\DataTypeTok}[1]{\textcolor[rgb]{0.13,0.29,0.53}{{#1}}}
\newcommand{\DecValTok}[1]{\textcolor[rgb]{0.00,0.00,0.81}{{#1}}}
\newcommand{\BaseNTok}[1]{\textcolor[rgb]{0.00,0.00,0.81}{{#1}}}
\newcommand{\FloatTok}[1]{\textcolor[rgb]{0.00,0.00,0.81}{{#1}}}
\newcommand{\ConstantTok}[1]{\textcolor[rgb]{0.00,0.00,0.00}{{#1}}}
\newcommand{\CharTok}[1]{\textcolor[rgb]{0.31,0.60,0.02}{{#1}}}
\newcommand{\SpecialCharTok}[1]{\textcolor[rgb]{0.00,0.00,0.00}{{#1}}}
\newcommand{\StringTok}[1]{\textcolor[rgb]{0.31,0.60,0.02}{{#1}}}
\newcommand{\VerbatimStringTok}[1]{\textcolor[rgb]{0.31,0.60,0.02}{{#1}}}
\newcommand{\SpecialStringTok}[1]{\textcolor[rgb]{0.31,0.60,0.02}{{#1}}}
\newcommand{\ImportTok}[1]{{#1}}
\newcommand{\CommentTok}[1]{\textcolor[rgb]{0.56,0.35,0.01}{\textit{{#1}}}}
\newcommand{\DocumentationTok}[1]{\textcolor[rgb]{0.56,0.35,0.01}{\textbf{\textit{{#1}}}}}
\newcommand{\AnnotationTok}[1]{\textcolor[rgb]{0.56,0.35,0.01}{\textbf{\textit{{#1}}}}}
\newcommand{\CommentVarTok}[1]{\textcolor[rgb]{0.56,0.35,0.01}{\textbf{\textit{{#1}}}}}
\newcommand{\OtherTok}[1]{\textcolor[rgb]{0.56,0.35,0.01}{{#1}}}
\newcommand{\FunctionTok}[1]{\textcolor[rgb]{0.00,0.00,0.00}{{#1}}}
\newcommand{\VariableTok}[1]{\textcolor[rgb]{0.00,0.00,0.00}{{#1}}}
\newcommand{\ControlFlowTok}[1]{\textcolor[rgb]{0.13,0.29,0.53}{\textbf{{#1}}}}
\newcommand{\OperatorTok}[1]{\textcolor[rgb]{0.81,0.36,0.00}{\textbf{{#1}}}}
\newcommand{\BuiltInTok}[1]{{#1}}
\newcommand{\ExtensionTok}[1]{{#1}}
\newcommand{\PreprocessorTok}[1]{\textcolor[rgb]{0.56,0.35,0.01}{\textit{{#1}}}}
\newcommand{\AttributeTok}[1]{\textcolor[rgb]{0.77,0.63,0.00}{{#1}}}
\newcommand{\RegionMarkerTok}[1]{{#1}}
\newcommand{\InformationTok}[1]{\textcolor[rgb]{0.56,0.35,0.01}{\textbf{\textit{{#1}}}}}
\newcommand{\WarningTok}[1]{\textcolor[rgb]{0.56,0.35,0.01}{\textbf{\textit{{#1}}}}}
\newcommand{\AlertTok}[1]{\textcolor[rgb]{0.94,0.16,0.16}{{#1}}}
\newcommand{\ErrorTok}[1]{\textcolor[rgb]{0.64,0.00,0.00}{\textbf{{#1}}}}
\newcommand{\NormalTok}[1]{{#1}}
\usepackage{longtable,booktabs}
\usepackage{graphicx,grffile}
\makeatletter
\def\maxwidth{\ifdim\Gin@nat@width>\linewidth\linewidth\else\Gin@nat@width\fi}
\def\maxheight{\ifdim\Gin@nat@height>\textheight\textheight\else\Gin@nat@height\fi}
\makeatother
% Scale images if necessary, so that they will not overflow the page
% margins by default, and it is still possible to overwrite the defaults
% using explicit options in \includegraphics[width, height, ...]{}
\setkeys{Gin}{width=\maxwidth,height=\maxheight,keepaspectratio}
\IfFileExists{parskip.sty}{%
\usepackage{parskip}
}{% else
\setlength{\parindent}{0pt}
\setlength{\parskip}{6pt plus 2pt minus 1pt}
}
\setlength{\emergencystretch}{3em}  % prevent overfull lines
\providecommand{\tightlist}{%
  \setlength{\itemsep}{0pt}\setlength{\parskip}{0pt}}
\setcounter{secnumdepth}{5}
% Redefines (sub)paragraphs to behave more like sections
\ifx\paragraph\undefined\else
\let\oldparagraph\paragraph
\renewcommand{\paragraph}[1]{\oldparagraph{#1}\mbox{}}
\fi
\ifx\subparagraph\undefined\else
\let\oldsubparagraph\subparagraph
\renewcommand{\subparagraph}[1]{\oldsubparagraph{#1}\mbox{}}
\fi

%%% Use protect on footnotes to avoid problems with footnotes in titles
\let\rmarkdownfootnote\footnote%
\def\footnote{\protect\rmarkdownfootnote}

%%% Change title format to be more compact
\usepackage{titling}

% Create subtitle command for use in maketitle
\newcommand{\subtitle}[1]{
  \posttitle{
    \begin{center}\large#1\end{center}
    }
}

\setlength{\droptitle}{-2em}

  \title{Reproducible Research Workflows with Snakemake and \texttt{R}}
    \pretitle{\vspace{\droptitle}\centering\huge}
  \posttitle{\par}
  \subtitle{An Extended Tutorial for Economists and Social Scientists}
  \author{Lachlan Deer \\ Julian Langer}
    \preauthor{\centering\large\emph}
  \postauthor{\par}
      \predate{\centering\large\emph}
  \postdate{\par}
    \date{2018-08-06}

\usepackage{booktabs}
\usepackage{amsthm}
\makeatletter
\def\thm@space@setup{%
  \thm@preskip=8pt plus 2pt minus 4pt
  \thm@postskip=\thm@preskip
}
\makeatother

\usepackage{amsthm}
\newtheorem{theorem}{Theorem}[chapter]
\newtheorem{lemma}{Lemma}[chapter]
\theoremstyle{definition}
\newtheorem{definition}{Definition}[chapter]
\newtheorem{corollary}{Corollary}[chapter]
\newtheorem{proposition}{Proposition}[chapter]
\theoremstyle{definition}
\newtheorem{example}{Example}[chapter]
\theoremstyle{definition}
\newtheorem{exercise}{Exercise}[chapter]
\theoremstyle{remark}
\newtheorem*{remark}{Remark}
\newtheorem*{solution}{Solution}
\begin{document}
\maketitle

{
\setcounter{tocdepth}{1}
\tableofcontents
}
\chapter{Prerequisites}\label{prerequisites}

This is a \emph{sample} book written in \textbf{Markdown}. You can use
anything that Pandoc's Markdown supports, e.g., a math equation
\(a^2 + b^2 = c^2\).

The \textbf{bookdown} package can be installed from CRAN or Github:

\chapter{Motivating \& Rationale}\label{intro}

\section{A Case for Reproducibility}\label{a-case-for-reproducibility}

\subsection{How far to go in the quest for
reproducibility?}\label{how-far-to-go-in-the-quest-for-reproducibility}

\section{\texorpdfstring{What is \texttt{Snakemake} \& Why Should you
use
it?}{What is Snakemake \& Why Should you use it?}}\label{what-is-snakemake-why-should-you-use-it}

\section{\texorpdfstring{Why \texttt{R}?}{Why R?}}\label{why-r}

\section{Working Example: Replicating Mankiw, Romer and Weil's 1992
QJE}\label{working-example-replicating-mankiw-romer-and-weils-1992-qje}

Throughout our tutorial we are going to use a running example to
illustrate the concepts we discuss.

\section{The way forward}\label{the-way-forward}

For the purpose of this tutorial we will focus on replicating the
following aspects of the MRW paper:\footnote{A complete replication
  using the concepts presented in this tutorial is available
  \textbf{here}}

\begin{itemize}
\tightlist
\item
  Regression Tables 1 and 2: Estimating the Textbook- and Augmented
  Solow Model
\item
  Figure 1: Unconditional Versus Conditional Convergence
\end{itemize}

To replicate these we will need to proceed as follows:

\begin{enumerate}
\def\labelenumi{\arabic{enumi}.}
\tightlist
\item
  Perform some data management

  \begin{itemize}
  \tightlist
  \item
    Prepare the data before we run regressions
  \end{itemize}
\item
  Do some analysis. For example, run regressions for:

  \begin{enumerate}
  \def\labelenumii{\arabic{enumii}.}
  \tightlist
  \item
    Different subsets of data
  \item
    Alternative econometric specifications
  \end{enumerate}
\item
  Turn the statistical output of the regressions into a tabular format
  that we can insert into a document
\item
  Construct a set of graphs
\item
  Integrate the tables and graphs into a paper and a set of slides
  (optional)
\end{enumerate}

We hope that these 5 steps look familiar - as they were designed to
represent a simplifed workflow for an applied economist or social
science researcher.

Before proceeding to understanding how to use Snakemake and R to
construct a reproducible workflow, the next chapter first takes a deeper
dive into the a protypical way to set up a research project on our
computer.

\subsection*{Exercise: Your own project's
steps}\label{exercise-your-own-projects-steps}
\addcontentsline{toc}{subsection}{Exercise: Your own project's steps}

Think about a project you are working on or have worked on in the past
(it may be a Bachelor or Master's thesis or a recent / active research
project). Does your project fit into the 5 steps we described above? If
not, what would you modify or add to our 5 steps? (Do you think this
would destroy the general principles we will encourage over the next
chapters?)

\chapter{Project Organization}\label{project-organization}

\section{Project Structure I: Separating Inputs and
Outputs}\label{project-structure-i-separating-inputs-and-outputs}

Structuring our project and the locations of files is an important
concept.

Let's look at the structure of our project's folder. Open a terminal and
change into this directory

\begin{Shaded}
\begin{Highlighting}[]
\KeywordTok{cd} \NormalTok{YOUR/PATH/TO/snakemake-econ-r-student}
\end{Highlighting}
\end{Shaded}

And list the subdirectories of the main directory

\begin{Shaded}
\begin{Highlighting}[]
\KeywordTok{ls} \NormalTok{-d */}
\end{Highlighting}
\end{Shaded}

We see the following folder structure

\begin{Shaded}
\begin{Highlighting}[]
\KeywordTok{./}
    \KeywordTok{|-} \NormalTok{src/}
    \KeywordTok{|-} \NormalTok{out/}
    \KeywordTok{|-} \NormalTok{log/}
    \KeywordTok{|-} \NormalTok{sandbox/}
\end{Highlighting}
\end{Shaded}

We recommend the following structure for any project:

\begin{itemize}
\tightlist
\item
  Root Folder
\item
  \texttt{src} folder for input files
\item
  \texttt{out} folder for output files
\item
  a \texttt{log} folder to store computer logs
\item
  a \texttt{sandbox} folder that gives us a `safe place' to develop new
  code
\end{itemize}

We discuss each of these in turn.

\subsection{The Root Folder}\label{the-root-folder}

TBD

\subsection{\texorpdfstring{The \texttt{src}
folder}{The src folder}}\label{the-src-folder}

TBD

\subsection{\texorpdfstring{The \texttt{out}
folder}{The out folder}}\label{the-out-folder}

TBD

\subsection{\texorpdfstring{The \texttt{log}
folder}{The log folder}}\label{the-log-folder}

\subsection{Exploring the Full Structure of the MRW Replication
Project}\label{exploring-the-full-structure-of-the-mrw-replication-project}

Now, let's look at all contents of this main projects directory:

\begin{Shaded}
\begin{Highlighting}[]
\KeywordTok{ls} \NormalTok{-F .}
\end{Highlighting}
\end{Shaded}

We see the following folder structure

\begin{Shaded}
\begin{Highlighting}[]
\KeywordTok{./}
    \KeywordTok{|-} \NormalTok{src/}
    \KeywordTok{|-} \NormalTok{out/}
    \KeywordTok{|-} \NormalTok{log/}
    \KeywordTok{|-} \NormalTok{sandbox/}
    \KeywordTok{|} \KeywordTok{README.md}
    \KeywordTok{|} \KeywordTok{Snakefile}
\end{Highlighting}
\end{Shaded}

Notice that there are no instances of: (i) scripts, (ii) files
containing content of the paper or slides (iii) something else we
haven't thought of yet Instead, there are only two files, a
\texttt{README.md} and a file called \texttt{Snakefile.}

TODO: explain these two files

\section{Project Structure II: Separating Logical Chunks of the
Project}\label{project-structure-ii-separating-logical-chunks-of-the-project}

As we have mentioned above, to keep our project's structure clean, we
want to keep all the computer code inside the \texttt{src} directory.
Let's have a look at the content of \texttt{src}.

\begin{Shaded}
\begin{Highlighting}[]
\KeywordTok{ls} \NormalTok{-F src/}
\end{Highlighting}
\end{Shaded}

We see the following output:

\begin{Shaded}
\begin{Highlighting}[]
\KeywordTok{./}
    \KeywordTok{|src/}
        \KeywordTok{|-} \NormalTok{data/}
        \KeywordTok{|-} \NormalTok{data-management/}
        \KeywordTok{|-} \NormalTok{data-specs/}
        \KeywordTok{|-} \NormalTok{analysis/}
        \KeywordTok{|-} \NormalTok{model-specs/}
        \KeywordTok{|-} \NormalTok{lib/}
        \KeywordTok{|-} \NormalTok{figures/}
        \KeywordTok{|-} \NormalTok{tables/}
\end{Highlighting}
\end{Shaded}

The type of content we expect in each file is:

TBD

\subsection{Exploring the Structure of the MRW Replication
Subdirectories}\label{exploring-the-structure-of-the-mrw-replication-subdirectories}

We begin our exploration of the project by looking at the folders that
appear to be related to the data. If we look inside the \texttt{data}
directory

\begin{Shaded}
\begin{Highlighting}[]
\KeywordTok{ls} \NormalTok{-F src/data/}
\end{Highlighting}
\end{Shaded}

\begin{Shaded}
\begin{Highlighting}[]
\KeywordTok{mrw.dta}
\end{Highlighting}
\end{Shaded}

That is, our \texttt{data/} directory contains the project's original
data set.

Note that in more extensive projects, the \texttt{data/} subfolder would
typically have more than one data set. For example:

\begin{Shaded}
\begin{Highlighting}[]
\KeywordTok{dataset1.dta}
\KeywordTok{dataset2.dta}
\KeywordTok{dataset3.csv}
\end{Highlighting}
\end{Shaded}

TBD - aside on file endings.

Further, your data folder may even contain further subdirectories that
organize data further

\begin{Shaded}
\begin{Highlighting}[]
\KeywordTok{./}
    \KeywordTok{|src/}
        \KeywordTok{|-} \NormalTok{data/}
            \KeywordTok{|-} \NormalTok{data-provider-a/}
                \KeywordTok{|-} \NormalTok{dataset1.csv}
                \KeywordTok{|-} \NormalTok{dataset2.csv}
            \KeywordTok{|-} \NormalTok{data-provider-b/}
                \KeywordTok{|-} \NormalTok{dataset3.txt}
                \KeywordTok{|-} \NormalTok{dataset4.txt}
\end{Highlighting}
\end{Shaded}

If we now turn to the \texttt{data-management} directory, we can explore
it's contents too:

\begin{Shaded}
\begin{Highlighting}[]
\KeywordTok{ls} \NormalTok{-F src/data-management/}
\end{Highlighting}
\end{Shaded}

\begin{Shaded}
\begin{Highlighting}[]
\KeywordTok{rename_variables.R}
\KeywordTok{gen_reg_vars.R}
\end{Highlighting}
\end{Shaded}

TODO:

\begin{itemize}
\tightlist
\item
  meaningful filenames
\item
  Note two different ways to name files
\end{itemize}

\subsection*{Exercise: Exploring the Remaining
Subdirectories}\label{exercise-exploring-the-remaining-subdirectories}
\addcontentsline{toc}{subsection}{Exercise: Exploring the Remaining
Subdirectories}

TBD

\section{Project Structure III: Separating Input Parameters from
Code}\label{project-structure-iii-separating-input-parameters-from-code}

Next we look at the somewhat mysteriously named \texttt{data-specs}
folder.

And if we explore the folder's contents:

\begin{Shaded}
\begin{Highlighting}[]
\KeywordTok{ls} \NormalTok{-F src/data-specs/}
\end{Highlighting}
\end{Shaded}

\begin{Shaded}
\begin{Highlighting}[]
\KeywordTok{subset_intermediate.json}
\KeywordTok{subset_nonoil.json}
\KeywordTok{subset_oecd.json}
\end{Highlighting}
\end{Shaded}

Again, the file names are somewhat meaningful on their own - they appear
to be some way of subsetting data (selecting some rows). If we look
inside one of these files:

\begin{Shaded}
\begin{Highlighting}[]
\KeywordTok{cat} \NormalTok{src/data-specs/subset_oecd.json}
\end{Highlighting}
\end{Shaded}

\begin{Shaded}
\begin{Highlighting}[]
\KeywordTok{\{}
    \StringTok{"KEEP_CONDITION"}\NormalTok{: }\StringTok{"oecd == 1"}
\KeywordTok{\}}
\end{Highlighting}
\end{Shaded}

We see an a variable \texttt{KEEP\_CONDITION} which is storing a string
\texttt{"oecd\ ==\ 1"}.

TBD: Why have we done this? See below.

\subsection{Exploring Parameter Separation in the MRW Replication
Project}\label{exploring-parameter-separation-in-the-mrw-replication-project}

\chapter{Initial Steps with
Snakemake}\label{initial-steps-with-snakemake}

\section{Starting a Research Project}\label{starting-a-research-project}

We are now ready to get started working with the code and data to build
a fully reproducible pipeline. In Chapter XX we described a simplified
research workflow to be:

\begin{enumerate}
\def\labelenumi{\arabic{enumi}.}
\tightlist
\item
  Perform some data management
\item
  Do some analysis
\item
  Turn the output of the analysis into a tabular format
\item
  Construct a set of graphs
\item
  Integrate the tables and graphs into a paper and a set of slides
  (optional)
\end{enumerate}

We are going to start at the beginning with data management.

Recall that we have the following files in our data management
subdirectory, \texttt{src/data-management}:

\begin{Shaded}
\begin{Highlighting}[]
\KeywordTok{rename_variables.R}
\KeywordTok{gen_reg_vars.R}
\end{Highlighting}
\end{Shaded}

We will need to run each of these scripts sequentially. First we want to
run the script \texttt{rename\_variables.R} to tidy up the variable
names in our data set. Second, \texttt{gen\_reg\_vars.R} will create the
some additional variables in our data that will be needed to run some
regressions in later steps. Over the next few sections we are going to
build up 2 \textbf{rules}, one for each file, that will execute these
scripts and deliver output.

\section{The Beginning of a
Snakefile}\label{the-beginning-of-a-snakefile}

We are going to put the collection of rules that build our project into
a file. We can then use the \texttt{Snakemake} to execute these rules
and build our project. The set of rules we want to construct are going
to go into the file called \texttt{Snakefile} - which is the name of a
file that Snakemake will look into by default to execite a project. Lets
open the file called \texttt{Snakefile} in the project's main directory.
When you open it it should look as follows:

\begin{Shaded}
\begin{Highlighting}[]
\CommentTok{# Main Workflow - SOME PROJECT}
\CommentTok{# Contributors: YOUR NAME(S)}
\ImportTok{import} \NormalTok{glob, os}
\CommentTok{# --- Variable Declarations ---- #}
\NormalTok{runR }\OperatorTok{=} \StringTok{"Rscript --no-save --no-restore --verbose"}
\NormalTok{logAll }\OperatorTok{=} \StringTok{"2>&1"}
\CommentTok{# --- Main Build Rules --- #}
\CommentTok{## To be constructed}
\end{Highlighting}
\end{Shaded}

Note that the amount of structure we have here is not totally necessary.
However, good structure will make understanding easier later. Let's go
through what we see. The first lines of code are comments, to help us
navigate a little and understand what we are looking at. The very first
line tells us that this is a project workflow, and then tells us what
the particular project is. The second line tells us who contributed to
this file. This can be useful so we know who to contact with questions.
You should do update the name of the project, and add your name to the
list of contributors. For us, the top 2 lines becomes:

\begin{Shaded}
\begin{Highlighting}[]
\CommentTok{# Main Workflow - Replicating MRW}
\CommentTok{# Contributors: @lachlandeer, @julianlanger}
\end{Highlighting}
\end{Shaded}

Next, we see the line:

\begin{Shaded}
\begin{Highlighting}[]
\ImportTok{import} \NormalTok{glob, os}
\end{Highlighting}
\end{Shaded}

What we are doing here is loading two additional libraries that will
help us later on with some functionality. Essentially, we want some
``helper'' functions to make the rest of our Snakemake process easier to
build. When we get to the point we need them, we will be more explicit.

The lines:

\begin{Shaded}
\begin{Highlighting}[]
\CommentTok{# --- Variable Declarations ---- #}
\NormalTok{runR }\OperatorTok{=} \StringTok{"Rscript --no-save --no-restore --verbose"}
\NormalTok{logAll }\OperatorTok{=} \StringTok{"2>&1"}
\end{Highlighting}
\end{Shaded}

are creating some shorthand notation for us. \texttt{runR} will be what
we want to use when we need to execute an \texttt{R} script. We declare
the shorthand so will not need to always write out the rather long
statement we have assigned to the value \texttt{runR.} \texttt{logAll}
is a code snippet that will log all the output from code execution and
all error messages to one file. This is useful when debugging errors in
our code - which we hope won't happen too often. The exact value here
(\texttt{"2\textgreater{}\&1"}) is specific to the R language.

The next few lines are:

\begin{Shaded}
\begin{Highlighting}[]
\CommentTok{# --- Main Build Rules --- #}
\CommentTok{## To be constructed}
\end{Highlighting}
\end{Shaded}

These are comments. We are using the
\texttt{\#\ -\/-\/-\ Something\ -\/-\/-\ \#} notation to break up the
code into logical blocks. It is in this block that we will assemble the
rules on which our project will be built.

Looking generally at the code, if you have some familarity with the
Python programming language you might recognise that this looks quite
some form of Python code. You are correct - Snakemake is written using
Python, so lots of what we do here will look like Python code. If you
are unfamiliar with Python, do not worry - there is nothing here that
will prohibit you from going further, and we will explain what is
happening as we go.

\chapter{Automatic Variables}\label{automatic-variables}

Content is TBD

\chapter{Getting Dependencies Right}\label{getting-dependencies-right}

Content is TBD

\chapter{Packrat: Managing Package
Dependencies}\label{packrat-managing-package-dependencies}

Content is TBD

\chapter{Pattern Rules}\label{pattern-rules}

Content is TBD

\chapter{Building The Entire Project}\label{building-the-entire-project}

Content is TBD

\chapter{Configuration Files for Tidy
Paths}\label{configuration-files-for-tidy-paths}

Content is TBD

\chapter{Subworkflows: Divide and
Conquer}\label{subworkflows-divide-and-conquer}

Content is TBD

\chapter{Building a Paper and Slides with
Rmarkdown}\label{building-a-paper-and-slides-with-rmarkdown}

Content is TBD

\chapter{Concluding Thoughts}\label{concluding-thoughts}

Content is TBD

\bibliography{references.bib}


\end{document}
