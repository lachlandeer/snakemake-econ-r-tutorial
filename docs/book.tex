\documentclass[]{book}
\usepackage{lmodern}
\usepackage{amssymb,amsmath}
\usepackage{ifxetex,ifluatex}
\usepackage{fixltx2e} % provides \textsubscript
\ifnum 0\ifxetex 1\fi\ifluatex 1\fi=0 % if pdftex
  \usepackage[T1]{fontenc}
  \usepackage[utf8]{inputenc}
\else % if luatex or xelatex
  \ifxetex
    \usepackage{mathspec}
  \else
    \usepackage{fontspec}
  \fi
  \defaultfontfeatures{Ligatures=TeX,Scale=MatchLowercase}
\fi
% use upquote if available, for straight quotes in verbatim environments
\IfFileExists{upquote.sty}{\usepackage{upquote}}{}
% use microtype if available
\IfFileExists{microtype.sty}{%
\usepackage{microtype}
\UseMicrotypeSet[protrusion]{basicmath} % disable protrusion for tt fonts
}{}
\usepackage[margin=1in]{geometry}
\usepackage{hyperref}
\hypersetup{unicode=true,
            pdftitle={Reproducible Research Workflows with Snakemake and R},
            pdfauthor={Lachlan Deer; Julian Langer},
            pdfborder={0 0 0},
            breaklinks=true}
\urlstyle{same}  % don't use monospace font for urls
\usepackage{natbib}
\bibliographystyle{apalike}
\usepackage{color}
\usepackage{fancyvrb}
\newcommand{\VerbBar}{|}
\newcommand{\VERB}{\Verb[commandchars=\\\{\}]}
\DefineVerbatimEnvironment{Highlighting}{Verbatim}{commandchars=\\\{\}}
% Add ',fontsize=\small' for more characters per line
\usepackage{framed}
\definecolor{shadecolor}{RGB}{248,248,248}
\newenvironment{Shaded}{\begin{snugshade}}{\end{snugshade}}
\newcommand{\KeywordTok}[1]{\textcolor[rgb]{0.13,0.29,0.53}{\textbf{{#1}}}}
\newcommand{\DataTypeTok}[1]{\textcolor[rgb]{0.13,0.29,0.53}{{#1}}}
\newcommand{\DecValTok}[1]{\textcolor[rgb]{0.00,0.00,0.81}{{#1}}}
\newcommand{\BaseNTok}[1]{\textcolor[rgb]{0.00,0.00,0.81}{{#1}}}
\newcommand{\FloatTok}[1]{\textcolor[rgb]{0.00,0.00,0.81}{{#1}}}
\newcommand{\ConstantTok}[1]{\textcolor[rgb]{0.00,0.00,0.00}{{#1}}}
\newcommand{\CharTok}[1]{\textcolor[rgb]{0.31,0.60,0.02}{{#1}}}
\newcommand{\SpecialCharTok}[1]{\textcolor[rgb]{0.00,0.00,0.00}{{#1}}}
\newcommand{\StringTok}[1]{\textcolor[rgb]{0.31,0.60,0.02}{{#1}}}
\newcommand{\VerbatimStringTok}[1]{\textcolor[rgb]{0.31,0.60,0.02}{{#1}}}
\newcommand{\SpecialStringTok}[1]{\textcolor[rgb]{0.31,0.60,0.02}{{#1}}}
\newcommand{\ImportTok}[1]{{#1}}
\newcommand{\CommentTok}[1]{\textcolor[rgb]{0.56,0.35,0.01}{\textit{{#1}}}}
\newcommand{\DocumentationTok}[1]{\textcolor[rgb]{0.56,0.35,0.01}{\textbf{\textit{{#1}}}}}
\newcommand{\AnnotationTok}[1]{\textcolor[rgb]{0.56,0.35,0.01}{\textbf{\textit{{#1}}}}}
\newcommand{\CommentVarTok}[1]{\textcolor[rgb]{0.56,0.35,0.01}{\textbf{\textit{{#1}}}}}
\newcommand{\OtherTok}[1]{\textcolor[rgb]{0.56,0.35,0.01}{{#1}}}
\newcommand{\FunctionTok}[1]{\textcolor[rgb]{0.00,0.00,0.00}{{#1}}}
\newcommand{\VariableTok}[1]{\textcolor[rgb]{0.00,0.00,0.00}{{#1}}}
\newcommand{\ControlFlowTok}[1]{\textcolor[rgb]{0.13,0.29,0.53}{\textbf{{#1}}}}
\newcommand{\OperatorTok}[1]{\textcolor[rgb]{0.81,0.36,0.00}{\textbf{{#1}}}}
\newcommand{\BuiltInTok}[1]{{#1}}
\newcommand{\ExtensionTok}[1]{{#1}}
\newcommand{\PreprocessorTok}[1]{\textcolor[rgb]{0.56,0.35,0.01}{\textit{{#1}}}}
\newcommand{\AttributeTok}[1]{\textcolor[rgb]{0.77,0.63,0.00}{{#1}}}
\newcommand{\RegionMarkerTok}[1]{{#1}}
\newcommand{\InformationTok}[1]{\textcolor[rgb]{0.56,0.35,0.01}{\textbf{\textit{{#1}}}}}
\newcommand{\WarningTok}[1]{\textcolor[rgb]{0.56,0.35,0.01}{\textbf{\textit{{#1}}}}}
\newcommand{\AlertTok}[1]{\textcolor[rgb]{0.94,0.16,0.16}{{#1}}}
\newcommand{\ErrorTok}[1]{\textcolor[rgb]{0.64,0.00,0.00}{\textbf{{#1}}}}
\newcommand{\NormalTok}[1]{{#1}}
\usepackage{longtable,booktabs}
\usepackage{graphicx,grffile}
\makeatletter
\def\maxwidth{\ifdim\Gin@nat@width>\linewidth\linewidth\else\Gin@nat@width\fi}
\def\maxheight{\ifdim\Gin@nat@height>\textheight\textheight\else\Gin@nat@height\fi}
\makeatother
% Scale images if necessary, so that they will not overflow the page
% margins by default, and it is still possible to overwrite the defaults
% using explicit options in \includegraphics[width, height, ...]{}
\setkeys{Gin}{width=\maxwidth,height=\maxheight,keepaspectratio}
\IfFileExists{parskip.sty}{%
\usepackage{parskip}
}{% else
\setlength{\parindent}{0pt}
\setlength{\parskip}{6pt plus 2pt minus 1pt}
}
\setlength{\emergencystretch}{3em}  % prevent overfull lines
\providecommand{\tightlist}{%
  \setlength{\itemsep}{0pt}\setlength{\parskip}{0pt}}
\setcounter{secnumdepth}{5}
% Redefines (sub)paragraphs to behave more like sections
\ifx\paragraph\undefined\else
\let\oldparagraph\paragraph
\renewcommand{\paragraph}[1]{\oldparagraph{#1}\mbox{}}
\fi
\ifx\subparagraph\undefined\else
\let\oldsubparagraph\subparagraph
\renewcommand{\subparagraph}[1]{\oldsubparagraph{#1}\mbox{}}
\fi

%%% Use protect on footnotes to avoid problems with footnotes in titles
\let\rmarkdownfootnote\footnote%
\def\footnote{\protect\rmarkdownfootnote}

%%% Change title format to be more compact
\usepackage{titling}

% Create subtitle command for use in maketitle
\newcommand{\subtitle}[1]{
  \posttitle{
    \begin{center}\large#1\end{center}
    }
}

\setlength{\droptitle}{-2em}

  \title{Reproducible Research Workflows with Snakemake and \texttt{R}}
    \pretitle{\vspace{\droptitle}\centering\huge}
  \posttitle{\par}
  \subtitle{An Extended Tutorial for Economists and Social Scientists}
  \author{Lachlan Deer \\ Julian Langer}
    \preauthor{\centering\large\emph}
  \postauthor{\par}
      \predate{\centering\large\emph}
  \postdate{\par}
    \date{2019-01-11}

\usepackage{booktabs}
\usepackage{amsthm}
\makeatletter
\def\thm@space@setup{%
  \thm@preskip=8pt plus 2pt minus 4pt
  \thm@postskip=\thm@preskip
}
\makeatother

\usepackage{amsthm}
\newtheorem{theorem}{Theorem}[chapter]
\newtheorem{lemma}{Lemma}[chapter]
\theoremstyle{definition}
\newtheorem{definition}{Definition}[chapter]
\newtheorem{corollary}{Corollary}[chapter]
\newtheorem{proposition}{Proposition}[chapter]
\theoremstyle{definition}
\newtheorem{example}{Example}[chapter]
\theoremstyle{definition}
\newtheorem{exercise}{Exercise}[chapter]
\theoremstyle{remark}
\newtheorem*{remark}{Remark}
\newtheorem*{solution}{Solution}
\begin{document}
\maketitle

{
\setcounter{tocdepth}{1}
\tableofcontents
}
\chapter*{Prerequisites}\label{prerequisites}
\addcontentsline{toc}{chapter}{Prerequisites}

This is a \emph{sample} book written in \textbf{Markdown}. You can use
anything that Pandoc's Markdown supports, e.g., a math equation
\(a^2 + b^2 = c^2\).

The \textbf{bookdown} package can be installed from CRAN or Github:

\chapter{Motivating \& Rationale}\label{intro}

\section{A Case for Reproducibility}\label{a-case-for-reproducibility}

\subsection{How far to go in the quest for
reproducibility?}\label{how-far-to-go-in-the-quest-for-reproducibility}

\section{\texorpdfstring{What is \texttt{Snakemake} \& Why Should you
use
it?}{What is Snakemake \& Why Should you use it?}}\label{what-is-snakemake-why-should-you-use-it}

\section{\texorpdfstring{Why \texttt{R}?}{Why R?}}\label{why-r}

\section{Working Example: Replicating Mankiw, Romer and Weil's 1992
QJE}\label{working-example-replicating-mankiw-romer-and-weils-1992-qje}

Throughout our tutorial we are going to use a running example to
illustrate the concepts we discuss.

\section{The way forward}\label{the-way-forward}

For the purpose of this tutorial we will focus on replicating the
following aspects of the MRW paper:\footnote{A complete replication
  using the concepts presented in this tutorial is available
  \textbf{here}}

\begin{itemize}
\tightlist
\item
  Regression Tables 1 and 2: Estimating the Textbook- and Augmented
  Solow Model
\item
  Figure 1: Unconditional Versus Conditional Convergence
\end{itemize}

To replicate these we will need to proceed as follows:

\begin{enumerate}
\def\labelenumi{\arabic{enumi}.}
\tightlist
\item
  Perform some data management

  \begin{itemize}
  \tightlist
  \item
    Prepare the data before we run regressions
  \end{itemize}
\item
  Do some analysis. For example, run regressions for:

  \begin{enumerate}
  \def\labelenumii{\arabic{enumii}.}
  \tightlist
  \item
    Different subsets of data
  \item
    Alternative econometric specifications
  \end{enumerate}
\item
  Turn the statistical output of the regressions into a tabular format
  that we can insert into a document
\item
  Construct a set of graphs
\item
  Integrate the tables and graphs into a paper and a set of slides
  (optional)
\end{enumerate}

We hope that these 5 steps look familiar - as they were designed to
represent a simplifed workflow for an applied economist or social
science researcher.

Before proceeding to understanding how to use Snakemake and R to
construct a reproducible workflow, the next chapter first takes a deeper
dive into the a protypical way to set up a research project on our
computer.

\subsection*{Exercise: Your own project's
steps}\label{exercise-your-own-projects-steps}
\addcontentsline{toc}{subsection}{Exercise: Your own project's steps}

Think about a project you are working on or have worked on in the past
(it may be a Bachelor or Master's thesis or a recent / active research
project). Does your project fit into the 5 steps we described above? If
not, what would you modify or add to our 5 steps? (Do you think this
would destroy the general principles we will encourage over the next
chapters?)

\chapter*{PART I}\label{part-i}
\addcontentsline{toc}{chapter}{PART I}

\chapter{Project Organization}\label{project-organization}

\section{Project Structure I: Separating Inputs and
Outputs}\label{project-structure-i-separating-inputs-and-outputs}

Structuring our project and the locations of files is an important
concept.

Let's look at the structure of our project's folder. Open a terminal and
change into this directory

\begin{Shaded}
\begin{Highlighting}[]
\KeywordTok{cd} \NormalTok{YOUR/PATH/TO/snakemake-econ-r-student}
\end{Highlighting}
\end{Shaded}

And list the subdirectories of the main directory

\begin{Shaded}
\begin{Highlighting}[]
\KeywordTok{ls} \NormalTok{-d */}
\end{Highlighting}
\end{Shaded}

We see the following folder structure

\begin{Shaded}
\begin{Highlighting}[]
\KeywordTok{./}
    \KeywordTok{|-} \NormalTok{src/}
    \KeywordTok{|-} \NormalTok{out/}
    \KeywordTok{|-} \NormalTok{log/}
    \KeywordTok{|-} \NormalTok{sandbox/}
\end{Highlighting}
\end{Shaded}

We recommend the following structure for any project:

\begin{itemize}
\tightlist
\item
  Root Folder
\item
  \texttt{src} folder for input files
\item
  \texttt{out} folder for output files
\item
  a \texttt{log} folder to store computer logs
\item
  a \texttt{sandbox} folder that gives us a `safe place' to develop new
  code
\end{itemize}

We discuss each of these in turn.

\subsection{The Root Folder}\label{the-root-folder}

TBD

\subsection{\texorpdfstring{The \texttt{src}
folder}{The src folder}}\label{the-src-folder}

TBD

\subsection{\texorpdfstring{The \texttt{out}
folder}{The out folder}}\label{the-out-folder}

TBD

\subsection{\texorpdfstring{The \texttt{log}
folder}{The log folder}}\label{the-log-folder}

\subsection{Exploring the Full Structure of the MRW Replication
Project}\label{exploring-the-full-structure-of-the-mrw-replication-project}

Now, let's look at all contents of this main projects directory:

\begin{Shaded}
\begin{Highlighting}[]
\KeywordTok{ls} \NormalTok{-F .}
\end{Highlighting}
\end{Shaded}

We see the following folder structure

\begin{Shaded}
\begin{Highlighting}[]
\KeywordTok{./}
    \KeywordTok{|-} \NormalTok{src/}
    \KeywordTok{|-} \NormalTok{out/}
    \KeywordTok{|-} \NormalTok{log/}
    \KeywordTok{|-} \NormalTok{sandbox/}
    \KeywordTok{|} \KeywordTok{README.md}
    \KeywordTok{|} \KeywordTok{Snakefile}
\end{Highlighting}
\end{Shaded}

Notice that there are no instances of: (i) scripts, (ii) files
containing content of the paper or slides (iii) something else we
haven't thought of yet Instead, there are only two files, a
\texttt{README.md} and a file called \texttt{Snakefile.}

TODO: explain these two files

\section{Project Structure II: Separating Logical Chunks of the
Project}\label{project-structure-ii-separating-logical-chunks-of-the-project}

As we have mentioned above, to keep our project's structure clean, we
want to keep all the computer code inside the \texttt{src} directory.
Let's have a look at the content of \texttt{src}.

\begin{Shaded}
\begin{Highlighting}[]
\KeywordTok{ls} \NormalTok{-F src/}
\end{Highlighting}
\end{Shaded}

We see the following output:

\begin{Shaded}
\begin{Highlighting}[]
\KeywordTok{./}
    \KeywordTok{|src/}
        \KeywordTok{|-} \NormalTok{data/}
        \KeywordTok{|-} \NormalTok{data-management/}
        \KeywordTok{|-} \NormalTok{data-specs/}
        \KeywordTok{|-} \NormalTok{analysis/}
        \KeywordTok{|-} \NormalTok{model-specs/}
        \KeywordTok{|-} \NormalTok{lib/}
        \KeywordTok{|-} \NormalTok{figures/}
        \KeywordTok{|-} \NormalTok{tables/}
\end{Highlighting}
\end{Shaded}

The type of content we expect in each file is:

TBD

\subsection{Exploring the Structure of the MRW Replication
Subdirectories}\label{exploring-the-structure-of-the-mrw-replication-subdirectories}

We begin our exploration of the project by looking at the folders that
appear to be related to the data. If we look inside the \texttt{data}
directory

\begin{Shaded}
\begin{Highlighting}[]
\KeywordTok{ls} \NormalTok{-F src/data/}
\end{Highlighting}
\end{Shaded}

\begin{Shaded}
\begin{Highlighting}[]
\KeywordTok{mrw.dta}
\end{Highlighting}
\end{Shaded}

That is, our \texttt{data/} directory contains the project's original
data set.

Note that in more extensive projects, the \texttt{data/} subfolder would
typically have more than one data set. For example:

\begin{Shaded}
\begin{Highlighting}[]
\KeywordTok{dataset1.dta}
\KeywordTok{dataset2.dta}
\KeywordTok{dataset3.csv}
\end{Highlighting}
\end{Shaded}

TBD - aside on file endings.

Further, your data folder may even contain further subdirectories that
organize data further

\begin{Shaded}
\begin{Highlighting}[]
\KeywordTok{./}
    \KeywordTok{|src/}
        \KeywordTok{|-} \NormalTok{data/}
            \KeywordTok{|-} \NormalTok{data-provider-a/}
                \KeywordTok{|-} \NormalTok{dataset1.csv}
                \KeywordTok{|-} \NormalTok{dataset2.csv}
            \KeywordTok{|-} \NormalTok{data-provider-b/}
                \KeywordTok{|-} \NormalTok{dataset3.txt}
                \KeywordTok{|-} \NormalTok{dataset4.txt}
\end{Highlighting}
\end{Shaded}

If we now turn to the \texttt{data-management} directory, we can explore
it's contents too:

\begin{Shaded}
\begin{Highlighting}[]
\KeywordTok{ls} \NormalTok{-F src/data-management/}
\end{Highlighting}
\end{Shaded}

\begin{Shaded}
\begin{Highlighting}[]
\KeywordTok{rename_variables.R}
\KeywordTok{gen_reg_vars.R}
\end{Highlighting}
\end{Shaded}

TODO:

\begin{itemize}
\tightlist
\item
  meaningful filenames
\item
  Note two different ways to name files
\end{itemize}

\subsection*{Exercise: Exploring the Remaining
Subdirectories}\label{exercise-exploring-the-remaining-subdirectories}
\addcontentsline{toc}{subsection}{Exercise: Exploring the Remaining
Subdirectories}

TBD

\section{Project Structure III: Separating Input Parameters from
Code}\label{project-structure-iii-separating-input-parameters-from-code}

Next we look at the somewhat mysteriously named \texttt{data-specs}
folder.

And if we explore the folder's contents:

\begin{Shaded}
\begin{Highlighting}[]
\KeywordTok{ls} \NormalTok{-F src/data-specs/}
\end{Highlighting}
\end{Shaded}

\begin{Shaded}
\begin{Highlighting}[]
\KeywordTok{subset_intermediate.json}
\KeywordTok{subset_nonoil.json}
\KeywordTok{subset_oecd.json}
\end{Highlighting}
\end{Shaded}

Again, the file names are somewhat meaningful on their own - they appear
to be some way of subsetting data (selecting some rows). If we look
inside one of these files:

\begin{Shaded}
\begin{Highlighting}[]
\KeywordTok{cat} \NormalTok{src/data-specs/subset_oecd.json}
\end{Highlighting}
\end{Shaded}

\begin{Shaded}
\begin{Highlighting}[]
\KeywordTok{\{}
    \StringTok{"KEEP_CONDITION"}\NormalTok{: }\StringTok{"oecd == 1"}
\KeywordTok{\}}
\end{Highlighting}
\end{Shaded}

We see an a variable \texttt{KEEP\_CONDITION} which is storing a string
\texttt{"oecd\ ==\ 1"}.

TBD: Why have we done this? See below.

\subsection{Exploring Parameter Separation in the MRW Replication
Project}\label{exploring-parameter-separation-in-the-mrw-replication-project}

\chapter{Initial Steps with
Snakemake}\label{initial-steps-with-snakemake}

\section{Starting a Research Project}\label{starting-a-research-project}

We are now ready to get started working with the code and data to build
a fully reproducible pipeline. In Chapter XX we described a simplified
research workflow to be:

\begin{enumerate}
\def\labelenumi{\arabic{enumi}.}
\tightlist
\item
  Perform some data management
\item
  Do some analysis
\item
  Turn the output of the analysis into a tabular format
\item
  Construct a set of graphs
\item
  Integrate the tables and graphs into a paper and a set of slides
  (optional)
\end{enumerate}

We are going to start at the beginning with data management.

Recall that we have the following files in our data management
subdirectory, \texttt{src/data-management}:

\begin{verbatim}
rename_variables.R
gen_reg_vars.R
\end{verbatim}

We will need to run each of these scripts sequentially. First we want to
run the script \texttt{rename\_variables.R} to tidy up the variable
names in our data set. Second, \texttt{gen\_reg\_vars.R} will create the
some additional variables in our data that will be needed to run some
regressions in later steps. Over the next few sections we are going to
build up 2 \textbf{rules}, one for each file, that will execute these
scripts and deliver output.

\section{The Beginning of a
Snakefile}\label{the-beginning-of-a-snakefile}

We are going to put the collection of rules that build our project into
a file. We can then use the \texttt{Snakemake} to execute these rules
and build our project. The set of rules we want to construct are going
to go into the file called \texttt{Snakefile} - which is the name of a
file that Snakemake will look into by default to execite a project. Lets
open the file called \texttt{Snakefile} in the project's main directory.
When you open it it should look as follows:

\begin{Shaded}
\begin{Highlighting}[]
\CommentTok{# Main Workflow - SOME PROJECT}
\CommentTok{#}
\CommentTok{# Contributors: YOUR NAME(S)}
\end{Highlighting}
\end{Shaded}

Note that the amount of structure we have here is not totally necessary.
However, good structure will make understanding easier later. Let's go
through what we see. The first lines of code are comments, to help us
navigate a little and understand what we are looking at. The very first
line tells us that this is a project workflow, and then tells us what
the particular project is. The second line tells us who contributed to
this file. This can be useful so we know who to contact with questions.
You should do update the name of the project, and add your name to the
list of contributors. For us, the top 2 lines becomes:

\begin{Shaded}
\begin{Highlighting}[]
\CommentTok{# Main Workflow - MRW Replication}
\CommentTok{#}
\CommentTok{# Contributors: @lachlandeer, @julianlanger}
\end{Highlighting}
\end{Shaded}

The next few lines are:

\begin{Shaded}
\begin{Highlighting}[]
\CommentTok{# --- Main Build Rules --- #}
\CommentTok{## To be constructed}
\end{Highlighting}
\end{Shaded}

These are more comments. We are using the
\texttt{\#\ -\/-\/-\ Something\ -\/-\/-\ \#} notation to break up the
code into logical blocks. It is in this block that we will assemble the
rules on which our project will be built.

\section{Rule Structure}\label{rule-structure}

A \texttt{Snakefile} is a collection of rules that together define the
order in which a project will be executed. In our \texttt{Snakefile} we
will start to assemble rules under the
\texttt{\#\ -\/-\/-\ Main\ Build\ Rules\ -\/-\/-\ \#} section to keep
things tidy. Each rule can be thought of as a recipe that combines
different \textbf{inputs}, such as data and and R script together to
produce one or more \textbf{output(s)}. The key components we are going
to use to construct a rule are:

\begin{enumerate}
\def\labelenumi{(\roman{enumi})}
\tightlist
\item
  a name for the rule,
\item
  the list of inputs
\item
  the list of outputs produced
\item
  a shell command that tells snakemake how to combine the inputs to
  produce a outputs.
\end{enumerate}

\texttt{Snakemake} expects these components to be provided in a
particular way so that it knows what to do with the information you
provided. We are going to specify rules in the following format:

\begin{Shaded}
\begin{Highlighting}[]
\KeywordTok{rule} \NormalTok{rule_name:}
    \KeywordTok{input}\NormalTok{:}
        \KeywordTok{input_name1} \NormalTok{= }\StringTok{"PATH/TO/input_one"}\NormalTok{,}
        \KeywordTok{input_name2} \NormalTok{= }\StringTok{"PATH/TO/input_two"}
    \KeywordTok{output}\NormalTok{:}
        \KeywordTok{output_name1} \NormalTok{= }\StringTok{"PATH/TO/SAVE/output_one"}\NormalTok{,}
        \KeywordTok{output_name2} \NormalTok{= }\StringTok{"PATH/TO/SAVE/output_two"}
    \KeywordTok{shell}\NormalTok{:}
        \StringTok{"HOW TO MIX IT ALL TOGETHER"}
\end{Highlighting}
\end{Shaded}

We can have as many inputs and outputs as we need to have per rule. Each
input and each output are given names, for example \texttt{input\_name1}
which take the value to the file path and name of the file. It is
important to wrap each of these paths into quotations, and to separate
each of the multiple inputs and outputs with a comma.

\section{Our First Rule}\label{our-first-rule}

\subsection{Constructing the Rule}\label{constructing-the-rule}

As mentioned above, we will start with the data management step.

First script to run is \texttt{rename\_variables.R} which is located in
the data management subdirectory. This is a simple script that renames
some variables for us to be easier to understand.

We can then start our snakemake script by adding this script as an
input:

\begin{Shaded}
\begin{Highlighting}[]
\KeywordTok{rule} \NormalTok{rule_name:}
    \KeywordTok{input}\NormalTok{:}
        \KeywordTok{script} \NormalTok{= }\StringTok{"src/data-management/rename_variables.R"}\NormalTok{,}
        \KeywordTok{input_name2} \NormalTok{= }\StringTok{"PATH/TO/input_two"}
    \KeywordTok{output}\NormalTok{:}
        \KeywordTok{output_name1} \NormalTok{= }\StringTok{"PATH/TO/SAVE/output_one"}\NormalTok{,}
        \KeywordTok{output_name2} \NormalTok{= }\StringTok{"PATH/TO/SAVE/output_two"}
    \KeywordTok{shell}\NormalTok{:}
        \StringTok{"HOW TO MIX IT ALL TOGETHER"}
\end{Highlighting}
\end{Shaded}

Next, we want to add any additional inputs and also specify any outputs
that the file produces. We have set up all R scripts in this example to
provide us with `help' so that we know what we might need to add. To
find out what inputs are required and what outputs are produced, we use
the \texttt{-\/-help} flag when calling the file with \texttt{R}:

\begin{Shaded}
\begin{Highlighting}[]
\NormalTok{$ }\KeywordTok{Rscript} \NormalTok{src/data-management/rename_variables.R --help}
\end{Highlighting}
\end{Shaded}

And the following output is produced:

\begin{verbatim}
Usage: src/data-management/rename_variables.R [options]


Options:
    -d CHARACTER, --data=CHARACTER
        stata dataset file name

    -o CHARACTER, --out=CHARACTER
        output file name [default = out.csv]

    -h, --help
        Show this help message and exit
\end{verbatim}

This suggests the script needs \ldots{}

We update our \texttt{rename\_variables} as:

\begin{Shaded}
\begin{Highlighting}[]
\KeywordTok{rule} \NormalTok{rename_vars:}
    \KeywordTok{input}\NormalTok{:}
        \KeywordTok{script} \NormalTok{= }\StringTok{"src/data-management/rename_variables.R"}\NormalTok{,}
        \KeywordTok{data}   \NormalTok{= }\StringTok{"src/data/mrw.dta"}
    \KeywordTok{output}\NormalTok{:}
        \KeywordTok{data} \NormalTok{= }\StringTok{"out/data/mrw_renamed.csv"}
    \KeywordTok{shell}\NormalTok{:}
        \StringTok{"HOW TO MIX IT ALL TOGETHER"}
\end{Highlighting}
\end{Shaded}

Next we provide a recipe telling snakemake how to mix the inputs to
create the outputs:

\begin{Shaded}
\begin{Highlighting}[]
\KeywordTok{rule} \NormalTok{rename_vars:}
    \KeywordTok{input}\NormalTok{:}
        \KeywordTok{script} \NormalTok{= }\StringTok{"src/data-management/rename_variables.R"}\NormalTok{,}
        \KeywordTok{data}   \NormalTok{= }\StringTok{"src/data/mrw.dta"}
    \KeywordTok{output}\NormalTok{:}
        \KeywordTok{data} \NormalTok{= }\StringTok{"out/data/mrw_renamed.csv"}
    \KeywordTok{shell}\NormalTok{:}
        \StringTok{"Rscript \{input.script\} \textbackslash{}}
\StringTok{            --data \{input.data\} \textbackslash{}}
\StringTok{            --out \{output.data\}"}
\end{Highlighting}
\end{Shaded}

We can now try and run snakemake to execute this rule \ldots{}

\begin{Shaded}
\begin{Highlighting}[]
\NormalTok{$ }\KeywordTok{snakemake}
\end{Highlighting}
\end{Shaded}

Stuff happens \ldots{}

\subsection{Analysing Output}\label{analysing-output}

We can look into our output directory to see if anything has happened:

\begin{Shaded}
\begin{Highlighting}[]
\NormalTok{$ }\KeywordTok{ls} \NormalTok{out/data/}
\end{Highlighting}
\end{Shaded}

which yields

\begin{verbatim}
mrw_renamed.csv
\end{verbatim}

Our file has been created as we expected.

Try and run snakemake again

\begin{Shaded}
\begin{Highlighting}[]
\NormalTok{$ }\KeywordTok{snakemake}
\end{Highlighting}
\end{Shaded}

and we see the following output:

\begin{verbatim}
Building DAG of jobs...
Nothing to be done.
\end{verbatim}

Why?

Snakemake provides the a \emph{summary} option which tells us what is
going on:

\begin{Shaded}
\begin{Highlighting}[]
\KeywordTok{snakemake} \NormalTok{--summary}
\end{Highlighting}
\end{Shaded}

The output is:

\begin{verbatim}
Building DAG of jobs...
output_file                       date                    rule      version  log-file(s)    status  plan
out/data/mrw_renamed.csv    Thu Jan 10 20:31:08 2019    rename_vars    -                     ok     no update
\end{verbatim}

Explain what this tells us \ldots{}

Suppose we updated one of the inputs \ldots{}

\begin{Shaded}
\begin{Highlighting}[]
\NormalTok{$ }\KeywordTok{touch} \NormalTok{src/data-management/rename_variables.R}
\end{Highlighting}
\end{Shaded}

and then look at the summary from snakemake:

\begin{Shaded}
\begin{Highlighting}[]
\NormalTok{$ }\KeywordTok{snakemake}
\end{Highlighting}
\end{Shaded}

\begin{verbatim}
Building DAG of jobs...
output_file                        date                    rule     version log-file(s)      status                plan
out/data/mrw_renamed.csv    Thu Jan 10 20:31:08 2019    rename_vars    -                updated input files update pending
\end{verbatim}

What this means?

Run snakemake to build the output:

\begin{verbatim}
snakemake
\end{verbatim}

\subsection*{Exercise: Deleting Output}\label{exercise-deleting-output}
\addcontentsline{toc}{subsection}{Exercise: Deleting Output}

Delete the output \texttt{out/data/mrw\_renamed.csv}. Run
\texttt{snakemake\ -\/-summary} and explain the output it produced.

\section{Creating a Second Rule}\label{creating-a-second-rule}

The second step in our data management is to create some variables we
will use in our regression analysis. The script
\texttt{gen\_reg\_vars.R} in the \texttt{src/data-management} folder
does this for us. We are going to build a rule called
\texttt{gen\_regression\_vars} to do this in Snakemake. Let's see what
the script expects to be passed:

\begin{Shaded}
\begin{Highlighting}[]
\NormalTok{$ }\KeywordTok{Rscript} \NormalTok{src/data-management/gen_reg_vars.R --help}
\end{Highlighting}
\end{Shaded}

\begin{verbatim}
Usage: src/data-management/gen_reg_vars.R [options]


Options:
    -d CHARACTER, --data=CHARACTER
        a csv file name

    -p CHARACTER, --param=CHARACTER
        a file name containing model parameters

    -o CHARACTER, --out=CHARACTER
        output file name [default = out.csv]

    -h, --help
        Show this help message and exit
\end{verbatim}

So we need to provide:

to provide the output \ldots{}

Let's create this rule:

\begin{Shaded}
\begin{Highlighting}[]
\KeywordTok{rule} \NormalTok{gen_regression_vars:}
    \KeywordTok{input}\NormalTok{:}
        \KeywordTok{script} \NormalTok{= }\StringTok{"src/data-management/gen_reg_vars.R"}\NormalTok{,}
        \KeywordTok{data}   \NormalTok{= }\StringTok{"out/data/mrw_renamed.csv"}\NormalTok{,}
        \KeywordTok{params} \NormalTok{= }\StringTok{"src/data-specs/param_solow.json"}\NormalTok{,}
    \KeywordTok{output}\NormalTok{:}
        \KeywordTok{data} \NormalTok{= }\StringTok{"out/data/mrw_complete.csv"}
    \KeywordTok{shell}\NormalTok{:}
        \StringTok{"Rscript \{input.script\} \textbackslash{}}
\StringTok{            --data \{input.data\} \textbackslash{}}
\StringTok{            --param \{input.params\} \textbackslash{}}
\StringTok{            --out \{output.data\}"}
\end{Highlighting}
\end{Shaded}

What will Snakemake want to do next? Let's use the summary option to see
\ldots{}

\begin{Shaded}
\begin{Highlighting}[]
\NormalTok{$ }\KeywordTok{snakemake} \NormalTok{--summary}
\end{Highlighting}
\end{Shaded}

\begin{verbatim}
Building DAG of jobs...
output_file                    date                     rule                  version   log-file(s) status     plan
out/data/mrw_complete.csv   -                           gen_regression_vars     -                   missing update pending
out/data/mrw_renamed.csv    Fri Jan 11 13:40:07 2019    rename_vars             -                   ok          no update
\end{verbatim}

Explain what this means

Let's run snakemake to build our new file:

\begin{Shaded}
\begin{Highlighting}[]
\NormalTok{$ }\KeywordTok{snakemake}
\end{Highlighting}
\end{Shaded}

When we look at our output directory:

\begin{Shaded}
\begin{Highlighting}[]
\NormalTok{$ }\KeywordTok{ls} \NormalTok{out/data/}
\end{Highlighting}
\end{Shaded}

\begin{verbatim}
mrw_complete.csv  mrw_renamed.csv
\end{verbatim}

So we see that our data set has been built.

\subsection*{Exercise: Creating Rules}\label{exercise-creating-rules}
\addcontentsline{toc}{subsection}{Exercise: Creating Rules}

The MRW paper estimates the Solow model for three subsets of data. You
need to create rules to do estimate the Solow model for each of these
data sets. The R script \texttt{src/analysis/estimate\_ols.R} will
estimate a OLS model for a given dataset when you provide the necessary
inputs.

\begin{enumerate}
\def\labelenumi{\arabic{enumi}.}
\tightlist
\item
  What inputs do you need to provide?
\item
  What outputs will be produced?
\item
  Write Snakemake rules to estimate the solow model for each subset of
  data.
\end{enumerate}

\section{Clean Rules}\label{clean-rules}

So far, we have built up our Snakefile to:

\begin{enumerate}
\def\labelenumi{\arabic{enumi}.}
\tightlist
\item
  Clean data
\item
  Run a regression model on different subsets of data
\end{enumerate}

As we continue to extend our Snakefile in the coming chapters we might
want to be able to delete all of the produced outputs, and see if we can
rebuild our project from the first step. One way to do this would be to
go to our terminal window and enter the following command each time:

\begin{Shaded}
\begin{Highlighting}[]
\NormalTok{$ }\KeywordTok{rm} \NormalTok{-rf out/*}
\end{Highlighting}
\end{Shaded}

Instead of doing this each time, we can integrate this \emph{cleaning}
of computer produced outputs into our Snakefile. We can create a rule
called \texttt{clean} that stores the shell command from above:

\begin{Shaded}
\begin{Highlighting}[]
\KeywordTok{rule} \NormalTok{clean:}
    \KeywordTok{shell}\NormalTok{:}
        \StringTok{"rm -rf out/*"}
\end{Highlighting}
\end{Shaded}

Note that this rule has no inputs or outputs.

To use this rule, we enter the following into our terminal:

\begin{Shaded}
\begin{Highlighting}[]
\NormalTok{$ }\KeywordTok{snakemake} \NormalTok{clean}
\end{Highlighting}
\end{Shaded}

Notice that to use the clean rule we had to call the rule name, clean,
explicitly.

Now if we look out the output of running the summary call with
snakemake, we see the following output:

\begin{Shaded}
\begin{Highlighting}[]
\KeywordTok{snakemake} \NormalTok{--summary}
\end{Highlighting}
\end{Shaded}

\begin{verbatim}
Building DAG of jobs...
output_file                                        date   rule  version log-file(s)     status  plan
out/analysis/model_solow_subset_intermediate.rds    -     inter        -                missing update pending
out/analysis/model_solow_subset_nonoil.rds          -     nonoil       -                missing update pending
out/analysis/model_solow_subset_oecd.rds            -     oecd         -                missing update pending
out/data/mrw_complete.csv                           -     gen_regression_vars           missing update pending
out/data/mrw_renamed.csv                            -     rename_vars                   missing update pending
\end{verbatim}

Which reveals snakemake's plan the next time its run will be to build
all outputs.

\subsection*{Exercise: Creating Cleaning
Rules}\label{exercise-creating-cleaning-rules}
\addcontentsline{toc}{subsection}{Exercise: Creating Cleaning Rules}

So far we have written a cleaning rule that deletes everything in the
\texttt{out/} directory. Construct rules that would separately clean the
\texttt{out/data/} and \texttt{out/analysis} subdirectories. Why might
we want to do this?

\chapter{Pattern Rules}\label{pattern-rules}

\section{Where we are now?}\label{where-we-are-now}

Your Snakefile should look something like this:

\begin{Shaded}
\begin{Highlighting}[]
\CommentTok{## Snakemake - MRW Replication}
\CommentTok{##}
\CommentTok{## @yourname}


\CommentTok{# --- Build Rules --- #}

\KeywordTok{rule} \NormalTok{solow_intermediate:}
    \KeywordTok{input}\NormalTok{:}
        \KeywordTok{script} \NormalTok{= }\StringTok{"src/analysis/estimate_ols_model.R"}\NormalTok{,}
        \KeywordTok{data}   \NormalTok{= }\StringTok{"out/data/mrw_complete.csv"}\NormalTok{,}
        \KeywordTok{model}  \NormalTok{= }\StringTok{"src/model-specs/model_solow.json"}\NormalTok{,}
        \KeywordTok{subset} \NormalTok{= }\StringTok{"src/data-specs/subset_intermediate.json"}
    \KeywordTok{output}\NormalTok{:}
        \KeywordTok{model_est} \NormalTok{= }\StringTok{"out/analysis/model_solow_subset_intermediate.rds"}\NormalTok{,}
    \KeywordTok{shell}\NormalTok{:}
        \StringTok{"Rscript \{input.script\} \textbackslash{}}
\StringTok{            --data \{input.data\} \textbackslash{}}
\StringTok{            --model \{input.model\} \textbackslash{}}
\StringTok{            --subset \{input.subset\} \textbackslash{}}
\StringTok{            --out \{output.model_est\}"}

\KeywordTok{rule} \NormalTok{solow_nonoil:}
    \KeywordTok{input}\NormalTok{:}
        \KeywordTok{script} \NormalTok{= }\StringTok{"src/analysis/estimate_ols_model.R"}\NormalTok{,}
        \KeywordTok{data}   \NormalTok{= }\StringTok{"out/data/mrw_complete.csv"}\NormalTok{,}
        \KeywordTok{model}  \NormalTok{= }\StringTok{"src/model-specs/model_solow.json"}\NormalTok{,}
        \KeywordTok{subset} \NormalTok{= }\StringTok{"src/data-specs/subset_nonoil.json"}
    \KeywordTok{output}\NormalTok{:}
        \KeywordTok{model_est} \NormalTok{= }\StringTok{"out/analysis/model_solow_subset_nonoil.rds"}\NormalTok{,}
    \KeywordTok{shell}\NormalTok{:}
        \StringTok{"Rscript \{input.script\} \textbackslash{}}
\StringTok{            --data \{input.data\} \textbackslash{}}
\StringTok{            --model \{input.model\} \textbackslash{}}
\StringTok{            --subset \{input.subset\} \textbackslash{}}
\StringTok{            --out \{output.model_est\}"}

\KeywordTok{rule} \NormalTok{solow_oecd:}
    \KeywordTok{input}\NormalTok{:}
        \KeywordTok{script} \NormalTok{= }\StringTok{"src/analysis/estimate_ols_model.R"}\NormalTok{,}
        \KeywordTok{data}   \NormalTok{= }\StringTok{"out/data/mrw_complete.csv"}\NormalTok{,}
        \KeywordTok{model}  \NormalTok{= }\StringTok{"src/model-specs/model_solow.json"}\NormalTok{,}
        \KeywordTok{subset} \NormalTok{= }\StringTok{"src/data-specs/subset_oecd.json"}
    \KeywordTok{output}\NormalTok{:}
        \KeywordTok{model_est} \NormalTok{= }\StringTok{"out/analysis/model_solow_subset_oecd.rds"}\NormalTok{,}
    \KeywordTok{shell}\NormalTok{:}
        \StringTok{"Rscript \{input.script\} \textbackslash{}}
\StringTok{            --data \{input.data\} \textbackslash{}}
\StringTok{            --model \{input.model\} \textbackslash{}}
\StringTok{            --subset \{input.subset\} \textbackslash{}}
\StringTok{            --out \{output.model_est\}"}

\KeywordTok{rule} \NormalTok{gen_regression_vars:}
    \KeywordTok{input}\NormalTok{:}
        \KeywordTok{script} \NormalTok{= }\StringTok{"src/data-management/gen_reg_vars.R"}\NormalTok{,}
        \KeywordTok{data}   \NormalTok{= }\StringTok{"out/data/mrw_renamed.csv"}\NormalTok{,}
        \KeywordTok{params} \NormalTok{= }\StringTok{"src/data-specs/param_solow.json"}\NormalTok{,}
    \KeywordTok{output}\NormalTok{:}
        \KeywordTok{data} \NormalTok{= }\StringTok{"out/data/mrw_complete.csv"}
    \KeywordTok{shell}\NormalTok{:}
        \StringTok{"Rscript \{input.script\} \textbackslash{}}
\StringTok{            --data \{input.data\} \textbackslash{}}
\StringTok{            --param \{input.params\} \textbackslash{}}
\StringTok{            --out \{output.data\}"}

\KeywordTok{rule} \NormalTok{rename_vars:}
    \KeywordTok{input}\NormalTok{:}
        \KeywordTok{script} \NormalTok{= }\StringTok{"src/data-management/rename_variables.R"}\NormalTok{,}
        \KeywordTok{data}   \NormalTok{= }\StringTok{"src/data/mrw.dta"}
    \KeywordTok{output}\NormalTok{:}
        \KeywordTok{data} \NormalTok{= }\StringTok{"out/data/mrw_renamed.csv"}
    \KeywordTok{shell}\NormalTok{:}
        \StringTok{"Rscript \{input.script\} \textbackslash{}}
\StringTok{            --data \{input.data\} \textbackslash{}}
\StringTok{            --out \{output.data\}"}

\CommentTok{# --- Clean Rules --- #}
\KeywordTok{rule} \NormalTok{clean:}
    \KeywordTok{shell}\NormalTok{:}
        \StringTok{"rm -rf out/*"}
\end{Highlighting}
\end{Shaded}

This is good progress, but if we look at the \emph{solow\_} rules we see
that there is quite a lot of duplication:

\begin{itemize}
\tightlist
\item
  Detail the duplication here
\end{itemize}

\section{Wildcards}\label{wildcards}

Ideally, we want our Snakefiles to feature the miniumum amount of
duplication possible (Why?). The three \emph{solow\_} rules can be
collapsed into one rule if can create a variable, for example
\texttt{iSubset}, that can iterate through the three .json files that
contain the subset filters. That is, we want to create a rule
\texttt{solow\_model} that can do the work that \texttt{solow\_nonoil},
\texttt{solow\_oecd} and \texttt{solow\_intermediate} currently do. In
Snakemake, these variables are called \emph{wildcards}.

The format of this rule will be:

\begin{Shaded}
\begin{Highlighting}[]
\KeywordTok{rule} \NormalTok{solow_model:}
    \KeywordTok{input}\NormalTok{:}
        \KeywordTok{script} \NormalTok{= }\StringTok{"src/analysis/estimate_ols_model.R"}\NormalTok{,}
        \KeywordTok{data}   \NormalTok{= }\StringTok{"out/data/mrw_complete.csv"}\NormalTok{,}
        \KeywordTok{model}  \NormalTok{= }\StringTok{"src/model-specs/model_solow.json"}\NormalTok{,}
        \KeywordTok{subset} \NormalTok{= }\StringTok{"src/data-specs/subset_\{iSubset\}.json"}
    \KeywordTok{output}\NormalTok{:}
        \KeywordTok{model_est} \NormalTok{= }\StringTok{"out/analysis/model_solow_\{iSubset\}.rds"}\NormalTok{,}
    \KeywordTok{shell}\NormalTok{:}
        \StringTok{"Rscript \{input.script\} \textbackslash{}}
\StringTok{            --data \{input.data\} \textbackslash{}}
\StringTok{            --model \{input.model\} \textbackslash{}}
\StringTok{            --subset \{input.subset\} \textbackslash{}}
\StringTok{            --out \{output.model_est\}"}
\end{Highlighting}
\end{Shaded}

We wrapped our wildcard \texttt{iSubset} in curly parentheses so that
Snakemake knows that we will want to substitute the name of one of the
subsets into this value. This is conceptually similar to what we have
done in our shell commands.

We can now try and run our updated Snakefile:

\begin{Shaded}
\begin{Highlighting}[]
\NormalTok{$ }\KeywordTok{snakemake}
\end{Highlighting}
\end{Shaded}

\begin{verbatim}
Building DAG of jobs...
WorkflowError:
Target rules may not contain wildcards. Please specify concrete files or a rule without wildcards.
\end{verbatim}

What has happened? Snakemake will not execute a rule that contains
wildcards.\footnote{Technically there is another problem here too.
  Snakemake doesn't know what values to substitute into
  \texttt{iSubset.} We focus on the Wildcard error because this is what
  the Snakemake error message mentions. By fixing this error, it turns
  out we also spell out what values to substitute into \texttt{iSubset}.}
What we will do is create another rule, \texttt{run\_solow} that will
not contain wildcards \ldots{}

\begin{Shaded}
\begin{Highlighting}[]

\KeywordTok{rule} \NormalTok{run_solow:}
    \KeywordTok{input}\NormalTok{:}
        \KeywordTok{nonoil} \NormalTok{= }\StringTok{"out/analysis/model_solow_nonoil.rds"}\NormalTok{,}
        \KeywordTok{oecd}   \NormalTok{= }\StringTok{"out/analysis/model_solow_oecd.rds"}\NormalTok{,}
        \KeywordTok{intermediate} \NormalTok{= }\StringTok{"out/analysis/model_solow_intermediate.rds"}

\KeywordTok{rule} \NormalTok{solow_model:}
    \KeywordTok{input}\NormalTok{:}
        \KeywordTok{script} \NormalTok{= }\StringTok{"src/analysis/estimate_ols_model.R"}\NormalTok{,}
        \KeywordTok{data}   \NormalTok{= }\StringTok{"out/data/mrw_complete.csv"}\NormalTok{,}
        \KeywordTok{model}  \NormalTok{= }\StringTok{"src/model-specs/model_solow.json"}\NormalTok{,}
        \KeywordTok{subset} \NormalTok{= }\StringTok{"src/data-specs/subset_\{iSubset\}.json"}
    \KeywordTok{output}\NormalTok{:}
        \KeywordTok{model_est} \NormalTok{= }\StringTok{"out/analysis/model_solow_\{iSubset\}.rds"}\NormalTok{,}
    \KeywordTok{shell}\NormalTok{:}
        \StringTok{"Rscript \{input.script\} \textbackslash{}}
\StringTok{            --data \{input.data\} \textbackslash{}}
\StringTok{            --model \{input.model\} \textbackslash{}}
\StringTok{            --subset \{input.subset\} \textbackslash{}}
\StringTok{            --out \{output.model_est\}"}
\end{Highlighting}
\end{Shaded}

Explain what happens here\ldots{}

There's more work we can do to reduce duplication. Look at the rule
\texttt{run\_solow}:

\begin{Shaded}
\begin{Highlighting}[]
\KeywordTok{rule} \NormalTok{run_solow:}
    \KeywordTok{input}\NormalTok{:}
        \KeywordTok{nonoil} \NormalTok{= }\StringTok{"out/analysis/model_solow_nonoil.rds"}\NormalTok{,}
        \KeywordTok{oecd}   \NormalTok{= }\StringTok{"out/analysis/model_solow_oecd.rds"}\NormalTok{,}
        \KeywordTok{intermediate} \NormalTok{= }\StringTok{"out/analysis/model_solow_intermediate.rds"}
\end{Highlighting}
\end{Shaded}

\section{\texorpdfstring{The \texttt{expand()}
function}{The expand() function}}\label{the-expand-function}

Each of these inputs listed above have similar structure, with only the
name of the subset of data we are using changing. We can use another
feature of Snakemake to simplify this rule. Snakemake has an
\texttt{expand()} function that can accept a wildcard and replace it
with a set of specified values in an iterative manner. In our case, we
want to use the expand function to accept the \texttt{\{iSubset\}}
wildcard, and replace it with the values `nonoil', `oecd' and
`intermediate' one at a time. To proceed we need to do two things:

\begin{enumerate}
\def\labelenumi{\arabic{enumi}.}
\tightlist
\item
  Create a list, DATA\_SUBSETS, that contains the values we want to
  iterate through - `nonoil', `oecd' and `intermediate'.
\item
  Use Snakemake's \texttt{expand()} function to iteratively replace
  \texttt{\{iSubset\}} with each value contained in the list
  \texttt{DATA\_SUBSET}
\end{enumerate}

Let's start with (1). We will use an area above our Snakemake rules to
store the \texttt{DATA\_SUBSET} variable:

\begin{Shaded}
\begin{Highlighting}[]
\KeywordTok{DATA_SUBSET} \NormalTok{= [}
                \StringTok{"oecd"}\NormalTok{,}
                \StringTok{"intermediate"}\NormalTok{,}
                \StringTok{"nonoil"}
                \NormalTok{]}

\CommentTok{# --- Build Rules --- #}
\KeywordTok{...}
\end{Highlighting}
\end{Shaded}

The capitalization of the list DATA\_SUBSET is not essential. We do it
to separate lists that we will iterate through from other parts of our
Snakefile. This means whenever we see a capitalized name, we know it is
a list that we want to iterate through.

Next, we update the rule \texttt{run\_solow} as follows:

\begin{Shaded}
\begin{Highlighting}[]
\KeywordTok{rule} \NormalTok{run_solow:}
    \KeywordTok{input}\NormalTok{:}
        \KeywordTok{expand}\NormalTok{(}\StringTok{"out/analysis/model_solow_\{iSubset\}.rds"}\NormalTok{,}
                    \KeywordTok{iSubset} \NormalTok{= DATA_SUBSET)}
\end{Highlighting}
\end{Shaded}

Now, we can clean our output folder with \texttt{\$\ snakemake\ clean}.
If we re-run snakemake we would expect everything to run. To see if this
is the case we can do a \emph{dry run}. A dry run will try go through
the Snakefile and print all the rules Snakemake wants to execute in
order.

\begin{Shaded}
\begin{Highlighting}[]
\NormalTok{$ }\KeywordTok{snakemake} \NormalTok{--dryrun}
\end{Highlighting}
\end{Shaded}

which yields the following plan:

\begin{verbatim}
Building DAG of jobs...
Job counts:
    count   jobs
    1   gen_regression_vars
    1   rename_vars
    1   run_solow
    3   solow_model
    6

[Fri Jan 11 17:00:32 2019]
rule rename_vars:
    input: src/data/mrw.dta, src/data-management/rename_variables.R
    output: out/data/mrw_renamed.csv
    jobid: 5


[Fri Jan 11 17:00:32 2019]
rule gen_regression_vars:
    input: out/data/mrw_renamed.csv, src/data-management/gen_reg_vars.R, src/data-specs/param_solow.json
    output: out/data/mrw_complete.csv
    jobid: 4


[Fri Jan 11 17:00:32 2019]
rule solow_model:
    input: src/data-specs/subset_nonoil.json, out/data/mrw_complete.csv, src/model-specs/model_solow.json, src/analysis/estimate_ols_model.R
    output: out/analysis/model_solow_nonoil.rds
    jobid: 1
    wildcards: iSubset=nonoil


[Fri Jan 11 17:00:32 2019]
rule solow_model:
    input: src/data-specs/subset_oecd.json, out/data/mrw_complete.csv, src/model-specs/model_solow.json, src/analysis/estimate_ols_model.R
    output: out/analysis/model_solow_oecd.rds
    jobid: 2
    wildcards: iSubset=oecd


[Fri Jan 11 17:00:32 2019]
rule solow_model:
    input: src/data-specs/subset_intermediate.json, out/data/mrw_complete.csv, src/model-specs/model_solow.json, src/analysis/estimate_ols_model.R
    output: out/analysis/model_solow_intermediate.rds
    jobid: 3
    wildcards: iSubset=intermediate


[Fri Jan 11 17:00:32 2019]
localrule run_solow:
    input: out/analysis/model_solow_nonoil.rds, out/analysis/model_solow_oecd.rds, out/analysis/model_solow_intermediate.rds
    jobid: 0

Job counts:
    count   jobs
    1   gen_regression_vars
    1   rename_vars
    1   run_solow
    3   solow_model
    6
\end{verbatim}

This looks like what we want to happen. Hence, we re-run snakemake to
produce all output:

\begin{Shaded}
\begin{Highlighting}[]
\NormalTok{$ }\KeywordTok{snakemake}
\end{Highlighting}
\end{Shaded}

\subsection*{Exercise: Exploring the expand() function
I}\label{exercise-exploring-the-expand-function-i}
\addcontentsline{toc}{subsection}{Exercise: Exploring the expand()
function I}

So far we have estimated the basic Solow model. MRW also estimate an
augmented version of the Solow model, adding human capital. The formula
required to estimate the augmented model is written up in
\texttt{src/model-specs/model\_aug\_solow.json}. Use the
\texttt{expand()} function together with the \texttt{estimate\_ols.R}
script to estimate the augmented solow model on each of the three data
subsets. The rule structures should look very similar to what we have
done so far.

\subsection*{Exercise: Exploring the expand function
II}\label{exercise-exploring-the-expand-function-ii}
\addcontentsline{toc}{subsection}{Exercise: Exploring the expand
function II}

The MRW paper contains three plots. Each of these plots use the subset
of `intermediate' countries. In the \texttt{src/figures/} subdirectory,
there are three scripts that reproduce each of the figures.\footnote{This
  is not entirely true, we are yet to figure out how to get the y-axis
  range from the original paper.} The scripts are written in such a way
that they accept exactly the same options. Using wildcards and the
expand function extend the Snakefile to construct each figure. Each
figure should be saved with the following name
`out/figures/SCRIPTNAME.pdf'

\section{Expanding Multiple
Wildcards}\label{expanding-multiple-wildcards}

The rules used to estimate the standard Solow model, and the augmented
Solow model have very similar structure:

\begin{Shaded}
\begin{Highlighting}[]
\KeywordTok{DATA_SUBSET} \NormalTok{= [}
                \StringTok{"oecd"}\NormalTok{,}
                \StringTok{"intermediate"}\NormalTok{,}
                \StringTok{"nonoil"}
                \NormalTok{]}

\CommentTok{# --- Build Rules --- #}
\KeywordTok{rule} \NormalTok{run_aug_solow:}
    \KeywordTok{input}\NormalTok{:}
        \KeywordTok{expand}\NormalTok{(}\StringTok{"out/analysis/model_aug_solow_\{iSubset\}.rds"}\NormalTok{,}
                    \KeywordTok{iSubset} \NormalTok{= DATA_SUBSET)}

\KeywordTok{rule} \NormalTok{aug_solow_model:}
    \KeywordTok{input}\NormalTok{:}
        \KeywordTok{script} \NormalTok{= }\StringTok{"src/analysis/estimate_ols_model.R"}\NormalTok{,}
        \KeywordTok{data}   \NormalTok{= }\StringTok{"out/data/mrw_complete.csv"}\NormalTok{,}
        \KeywordTok{model}  \NormalTok{= }\StringTok{"src/model-specs/model_aug_solow.json"}\NormalTok{,}
        \KeywordTok{subset} \NormalTok{= }\StringTok{"src/data-specs/subset_\{iSubset\}.json"}
    \KeywordTok{output}\NormalTok{:}
        \KeywordTok{model_est} \NormalTok{= }\StringTok{"out/analysis/model_aug_solow_\{iSubset\}.rds"}\NormalTok{,}
    \KeywordTok{shell}\NormalTok{:}
        \StringTok{"Rscript \{input.script\} \textbackslash{}}
\StringTok{            --data \{input.data\} \textbackslash{}}
\StringTok{            --model \{input.model\} \textbackslash{}}
\StringTok{            --subset \{input.subset\} \textbackslash{}}
\StringTok{            --out \{output.model_est\}"}

\KeywordTok{rule} \NormalTok{run_solow:}
    \KeywordTok{input}\NormalTok{:}
        \KeywordTok{expand}\NormalTok{(}\StringTok{"out/analysis/model_solow_\{iSubset\}.rds"}\NormalTok{,}
                    \KeywordTok{iSubset} \NormalTok{= DATA_SUBSET)}

\KeywordTok{rule} \NormalTok{solow_model:}
    \KeywordTok{input}\NormalTok{:}
        \KeywordTok{script} \NormalTok{= }\StringTok{"src/analysis/estimate_ols_model.R"}\NormalTok{,}
        \KeywordTok{data}   \NormalTok{= }\StringTok{"out/data/mrw_complete.csv"}\NormalTok{,}
        \KeywordTok{model}  \NormalTok{= }\StringTok{"src/model-specs/model_solow.json"}\NormalTok{,}
        \KeywordTok{subset} \NormalTok{= }\StringTok{"src/data-specs/subset_\{iSubset\}.json"}
    \KeywordTok{output}\NormalTok{:}
        \KeywordTok{model_est} \NormalTok{= }\StringTok{"out/analysis/model_solow_\{iSubset\}.rds"}\NormalTok{,}
    \KeywordTok{shell}\NormalTok{:}
        \StringTok{"Rscript \{input.script\} \textbackslash{}}
\StringTok{            --data \{input.data\} \textbackslash{}}
\StringTok{            --model \{input.model\} \textbackslash{}}
\StringTok{            --subset \{input.subset\} \textbackslash{}}
\StringTok{            --out \{output.model_est\}"}
\end{Highlighting}
\end{Shaded}

\chapter{Wildcards}\label{wildcards-1}

Content is TBD

\chapter{Adding Parameters}\label{adding-parameters}

Content is TBD

\chapter{\texorpdfstring{An \texttt{all}
Rule}{An all Rule}}\label{an-all-rule}

Content is TBD

\chapter{Logging Output and Errors}\label{logging-output-and-errors}

Content is TBD

\chapter{Self Documenting Help}\label{self-documenting-help}

Content is TBD

\chapter{Config Files}\label{config-files}

Content is TBD

\chapter{Subworkflows: Divide and
Conquer}\label{subworkflows-divide-and-conquer}

Content is TBD

\chapter*{PART II}\label{part-ii}
\addcontentsline{toc}{chapter}{PART II}

\chapter{Reproducible Articles with
Rmd}\label{reproducible-articles-with-rmd}

Content is TBD

\chapter{Reproducible Slides with
Rmd}\label{reproducible-slides-with-rmd}

Content TBD

\chapter{Package Dependencies with
Packrat}\label{package-dependencies-with-packrat}

\chapter{Containers}\label{containers}

\bibliography{references.bib}


\end{document}
